\section{Block Packed Storage}\label{sec: memory layout}

Let \(\Lmatrix_n\) be a triangular matrix of size \(kn\) and define
\[
	g(r):= \frac{r(r+1)}2 = \sum_{j=1}^r j.
\]
We can then represent \(\Lmatrix_n\) using full \(k\times k\) block matrices
\(L_i\) in the following way
\[
	\Lmatrix_n = \begin{pmatrix}
		L_0 & \\
		L_1 & L_2 \\
		L_3 & L_4 & L_5 \\
		\vdots & & & \ddots \\
		L_{g(n-1)} & & \dots & & L_{g(n) -1}
	\end{pmatrix}
\]
where all the matrices
\[
	L_0,L_2,L_5,\dots, L_{\frac{n(n+1)}2-1} = L_{g(1)-1}, \dots, L_{g(n)-1}
\]
have zeros above their diagonal, while the other matrices are full matrices.

If we now view \(Z_{n+1}\) as a \(k\) dimensional random vector we want to
simulate in bulk, we can still use Algorithm~\ref{algo: sim Z_{n+1}}, if we
use
\[
	c_n = \begin{pmatrix}
		C_0\\
		\vdots\\
		C_{n-1}	
	\end{pmatrix} \in \real^{kn\times k}
\]
where \(C_r\in\real^{k\times k}\) are block matrices representing the covariance
of the old random vectors \(Z_i\) with \(Z_{n+1}\). And
\[
	\sigma_n^2 \in \real^{k\times k}
\]
would be the covariance matrix of \(Z_{n+1}\) itself. In order to apply
Algorithm~\ref{algo: sim Z_{n+1}}, we only need to reinterpret the square root in
line~\ref{algo-line: sqrt} as a cholesky decomposition.

{
	\newcommand{\xc}{{\color{red} C}}
	\newcommand{\xL}{{\color{violet} L}}
	\newcommand{\xl}{{\color{teal} \Gamma}}
	\begin{algorithm}
		\caption{
			\((\Lmatrix_n^{-1}c_n)^T=c_n^T (\Lmatrix_n^T)^{-1}\) stored as new row in \(\Lmatrix_{n+1}\).
			Can be in-place if \(c_n^T\) is stored in this new row before usage.
		}	
		\label{algo: solve linear equation}
		\For(in parallel){\(r = 0,\dots,n-1\)}{
			\(\xc \leftarrow C_r^T\)\tcp*[f]{red cache, \(L_{g(n)+r}\) if in-place}\;
			\For{\(i = 0,\dots, r-1\)}{
				{
					\color{lightgray}
				wait for \(i\le\) finished\_r\tcp*[f]{parallelization safety}\label{algo-line: parallel safety check}\;
				}
				\(\xL \leftarrow L_{g(r)+i}\)\tcp*[f]{violet cache}\;
				\(\xl \leftarrow L_{g(n)+i}\)\tcp*[f]{teal cache}\;
				\(\xc = \xc - \xl\xL^T\)
			}
			\(\xL \leftarrow L_{g(r)+r}\)\tcp*[f]{
				\(=L_{g(r+1) -1}\) i.e. on the diagonal}\;
			\(\xc = \xc/\xL^T\)\;
			\(L_{g(n)+r} \leftarrow \xc\)\;
			{
				\color{lightgray}
				acquire lock: finished\_r\;
				finished\_r \(\leftarrow r\)\tcp*[f]{due to line~\ref{algo-line: parallel safety check} equivalent to increment}\;
				release lock: finished\_r\;
			}
		}
	\end{algorithm}
}
All of the caches in Algorithm~\ref{algo: solve linear equation} are of size
\(k\times k\).  Our total cache needs are therefore \(3k^2\). To reimplement the
remaining steps of Algorithm~\ref{algo: sim Z_{n+1}} we use Algorithm~\ref{algo:
calc lambda and simulate Z_{n+1}}. As the blue and violet cache only require
\(k\times 1\) space, it requires less than \(3k^2\) cache space for \(k\ge 2\).

{
	\newcommand{\xS}{{\color{red} S}}
	\newcommand{\xl}{{\color{teal} \Gamma}}
	\newcommand{\xY}{{\color{violet} Y}}
	\newcommand{\xZ}{{\color{blue} Z}}
	\begin{algorithm}
		\caption{Calculation of \(\lambda_n\) and Simulation of \(Z_{n+1}\)}	
		\label{algo: calc lambda and simulate Z_{n+1}}
		\(\xS \leftarrow \sigma_n\)\tcp*[f]{red cache}\;
		\(\xZ \leftarrow 0_{k\times 1}\)\tcp*[f]{blue cache}\;
		\For{\(i = 0,\dots,n-1\)}{
			\(\xl \leftarrow L_{g(n)+i}\)\tcp*[f]{teal cache}\;
			\(\xS = \xS - \xl\xl^T\)\;
			\(\xY \leftarrow Y_i\)\tcp*[f]{violet cache}\;
			\(\xZ = \xZ + \xl^T \xY\)\;
		}
		\(L_{g(n) + n} \leftarrow \chol(\xS)\)\tcp*[f]{\(=L_{g(n+1)-1}\) i.e. on the diagonal}\;
		\KwRet{\(\xZ\)}
	\end{algorithm}
}
